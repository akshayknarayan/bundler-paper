\section{Evaluation}\label{s:eval}

%We have seen in Figure~\ref{fig:design:shift-bottleneck} that \name can move the network queues to the \inbox to gain control over scheduling. 
%Given that this is possible, what benefits can \name achieve, and where do they come from?
Given \name's ability to move the in-network queues to the \inbox (as shown earlier in Figure~\ref{fig:design:shift-bottleneck}), we now explore:
\begin{enumerate}[leftmargin=15pt]
    \item Where do \name's performance benefits come from? We discuss this in the context of improving the flow completion times of \name's component flows. (\S\ref{s:eval:fct})
    \item Do \name's performance benefits hold across different scenarios? (\S\ref{s:robust:cross})
    \item Can \name work with different congestion control algorithms (\S\ref{s:eval:cc})?
    \item Are \name's core ideas still applicable with other design decisions? (\S\ref{s:eval:proxy})
    \item Is \name's heuristic (\S\ref{s:queue-ctl:ecmp}) for detecting imbalanced multipath scenarios robust? (\S\ref{s:eval:ecmp})  
    \item Can \name effectively control the queues on real Internet paths? (\S\ref{s:eval:realworld})
\end{enumerate}

\subsection{Experimental Setup}\label{s:eval:setup}

We use network emulation via mahimahi~\cite{mahimahi} to evaluate our implementation of \name in a controlled setting; we present results on real Internet paths in \S\ref{s:eval:realworld}.
There are three $8$-core Ubuntu 18.04 machines in our emulated setup: (1) runs a sender, (2) runs a \inbox, and (3) runs both a \outbox and a receiver.
% The \outbox runs on the same machine as the receiver.
We disable both TCP segmentation offload (TSO) and generic receive offload (GRO) as they would change the packet headers in between the \inbox and \outbox, which would cause inconsistent epoch boundary identification between the two boxes.
% Since our \outbox implementation uses \texttt{libpcap}, generic receive offload (GRO) would change the packets before they are delivered to the \outbox, which would cause inconsistent epoch boundary identification between the two boxes. We, therefore, disable TCP segmentation offload and generic receive offload.
Nevertheless, throughout our experiments CPU utilization on the machines remained below $10$\%.

Unless otherwise specified, we emulate the following scenario.
A many-threaded client generates requests from a request size CDF drawn from an Internet core router~\cite{caida-dataset} and assigns them to one of $200$ server processes.
The workload is heavy-tailed: 97.6\% of requests are 10KB or shorter, and the largest 0.002\% of requests are between $5$MB and $100$MB.
Each server then sends the requested amount of data to the client and we measure the flow completion time of each such request. 
The link bandwidth at the mahimahi link is set to 96Mbps, and the RTT is set to 50ms. The requests result in an offered load of 84Mbps. 

The endhost runs Cubic~\cite{cubic}, and the \inbox runs Copa~\cite{copa} (we test other schemes in \S\ref{s:eval:cc}) with Nimbus~\cite{nimbus-arxiv} for cross traffic detection. 
Each experiment is comprised of 1,000,000 requests sampled from this distribution, across 10 runs each with a different random seed.
%\an{new, short version of ``applying different scheduling policies''}:
The \inbox schedules traffic using stochastic fair queueing~\cite{sfq} in our experiments. 


\subsection{Understanding Performance Benefits}\label{s:eval:fct}

We first present results for a simplified scenario without any cross-traffic, \ie all traffic traversing through the network is generated by the same customer and is, therefore, part of the same bundle. 
This scenario highlights the benefits of using \name when the congestion on the bottleneck link in the network is self-inflicted. We explore the effects of congestion due to other cross-traffic in \S\ref{s:robust:cross}.

\begin{figure}
    \centering
\begin{knitrout}
\definecolor{shadecolor}{rgb}{0.969, 0.969, 0.969}\color{fgcolor}
\includegraphics[width=\maxwidth]{figure/eval:best-1} 

\end{knitrout}
    \caption{\name achieves 33\% lower median slowdown. Note the differing axis scales. For both \name and Optimal, performance benefits come from preventing short flows from queueing behind long ones.}
    \label{fig:eval:best}
\end{figure}
\newcommand{\overviewBenefitsBaselineMedian}{1.62\xspace}
\newcommand{\overviewBenefitsBaselineTail}{10.77\xspace}
\newcommand{\overviewBenefitsBundlerMedian}{1.08\xspace}
\newcommand{\overviewBenefitsBundlerTail}{9.84\xspace}
\newcommand{\overviewBenefitsOptimalMedian}{1.08\xspace}
\newcommand{\overviewBenefitsOptimalTail}{4.46\xspace}
\newcommand{\overviewBenefitsBundlerMedianImprovement}{33\%\xspace}

\newcommand{\baseline}{Status Quo\xspace}
\newcommand{\optimal}{In-Network\xspace}

\Para{Using \name for fair queueing}
In this section, we evaluate the benefits provided by doing fair queuing at the \name, and use median slowdown as our metric, where the ``slowdown'' of a request is its completion time divided by what its completion time would have been in an unloaded network. A slowdown of $1$ is optimal, and lower numbers represent better performance.

We evaluate three configurations: 
(i) The ``\baseline'' configuration represents the status quo: the \inbox simply forwards packets as it receives them, and the mahimahi bottleneck uses FIFO scheduling.
(ii) The ``\optimal'' configuration deploys fair queueing
at the mahimahi bottleneck\footnote{
We implement this scheme by modifying mahimahi (our patch comprises $171$ lines of C++) to add a packet-level fair-queueing scheduler to the bottleneck link.}. 
Recall from \S\ref{s:intro} that this configuration is not deployable.
%: it would force all customers --- who may desire diverging scheduling policies --- to use the same scheduler.
(iii) The default \name configuration, that uses stochastic fair queueing~\cite{sfq} scheduling policy at the \inbox, and (iv) Using \name with FIFO (without exploiting scheduling opportunity).
%\radhika{flip the order for in-network and \name in the figure to be consistent with the above order.}

Figure~\ref{fig:eval:best} presents our results. 
The median slowdown (across all flow sizes) decreases from \overviewBenefitsBaselineMedian 
for Baseline to \overviewBenefitsBundlerMedian 
with \name, \overviewBenefitsBundlerMedianImprovement
lower. 
\optimal's median slowdown is a further 15\% lower then \name: \overviewBenefitsOptimalMedian.
Meanwhile, in the tail, \name's $99\%$ile slowdown is \overviewBenefitsBundlerTail, which is 48\% lower than the \baseline's \overviewBenefitsBaselineTail. \optimal's $99\%$ile slowdown is \overviewBenefitsOptimalTail.

\paragrapha{Using \name for other policies} We additionally evaluated other scheduling and queue management policies with \name. We omit detailed results for brevity, and present a few highlights. With FQ-CoDel~\cite{fq-codel}, \name can achieve 97\% lower median end-to-end RTTs and 89\% lower 99\%ile RTTs.  By strictly prioritizing one traffic class over another, \name results in 65\% lower median FCTs for the higher-priority class. 

\cut{
\subsection{Applying Different Scheduling Policies}\label{s:eval:policies}
%Per \S\ref{s:impl}, our \name implementation can implement any scheduling discipline available in Linux. 

\an{Reduce this subsection to ~one paragraph and roll into 7.2}

In addition to improving flow completion times, \name can achieve low packet delay, perform strict prioritization, and rate fairness.

\paragrapha{Achieving Improved Flow Completion Times} \S\ref{s:eval:fct} shows how enabling SFQ at the \name improves the median slowdown by \overviewBenefitsBundlerMedianImprovement.

\paragrapha{Achieving Low Packet Delays}
We enable CoDel~\cite{fq-codel} at the \inbox to lower the packet delays, and test it for a single large backlogged flow using the setup described in \S\ref{s:eval}.
CoDel adds ECN marks to packets in fair-queue buckets which exceed a queue length threshold. 
As a result, endhosts cut their windows earlier, thus reducing their self-inflicted delay within their fair-queue bucket.
We measure the resulting distribution of RTTs seen by the endhost connections with \name and \baseline in Table~\ref{t:eval:codel}.
\begin{figure}
    \centering
\begin{knitrout}
\definecolor{shadecolor}{rgb}{0.969, 0.969, 0.969}\color{fgcolor}
\includegraphics[width=\maxwidth]{figure/eval:lowdelays-1} 

\end{knitrout}
    \caption{We configure \name with the fq-codel scheduling policy to achieve low end-to-end queueing delays.}
    \label{fig:eval:lowdelays}
\end{figure}

As is expected from CoDel, \name in this experiment, achieves \delaysImprovement lower median packet delay than \baseline.

\paragrapha{Strict Prioritization}\label{s:eval:strictprio}
We uniformly divide the web request distribution described in \S\ref{s:eval:setup} into two equally sized classes, one of which is given a higher priority over the other. 
%\radhika{what's the division ratio?}\an{specified equal ratio}  
The results are presented in Table~\ref{t:eval:prio}.
%of traffic transitting \name over a lower-priority class. 

\begin{center}
\begin{tabular}{c|c|c}
Scheme     &  Median                                    &  $99$\%ile                                 \\
\hline
Bundler    &  1.07  &  10.52  \\
\baseline  &  3.07  &  201.72
    \label{fig:eval:strict-prio}
\end{tabular}
\end{center}


Using a priority scheduler (we use the \texttt{pfifo\_fast} qdisc) at \inbox improves the flow completion times for the higher-priority class compared to \baseline.
Furthermore, prioritization achieves \strictPrioTailImprovementOverFq lower $99$\%ile FCT for the higher priority traffic class, when compared to using fair queueing.

\begin{figure}
    \centering
\begin{knitrout}
\definecolor{shadecolor}{rgb}{0.969, 0.969, 0.969}\color{fgcolor}
\includegraphics[width=\maxwidth]{figure/eval:waterfall-1} 

\end{knitrout}
    \caption{\name with SFQ achieves fair and stable rates.}
    \label{fig:eval:waterfall}
\end{figure}

\paragrapha{Rate Fairness and Stability}\label{s:eval:waterfall}
We next use our default SFQ scheduler to achieve fairness and rate stability. We start three backlogged flows at different times (0s, 15s, and 30s). Figure~\ref{fig:eval:waterfall} shows that \name converges to fair and stable rates faster than the \baseline.

\begin{figure}
    \label{fig:eval:video}
    \centering
    \begin{subfigure}[b]{0.5\textwidth}
        \includegraphics[width=\textwidth]{figure/nobundle-4k-video}
        \caption{Without \name}\label{fig:eval:video:nobundle}
    \end{subfigure}
    \begin{subfigure}[b]{0.5\textwidth}
        \includegraphics[width=\textwidth]{figure/bundle-4k-video}
        \caption{With \name}\label{fig:eval:video:bundle}
    \end{subfigure}
    \caption{Without \name, the video traffic experiences highly variable throughput, which prevents the ABR algorithm from realizing it could sustain a higher bitrate. \name helps the video flow to quickly converge to the fair rate and stay there, which allows the ABR algorithm to choose the maximum sustainable bitrate.}
\end{figure}

\paragrapha{Rate Stability}\label{s:eval:ratestable}
\fc{come back to this}
In Figure~\ref{fig:eval:video}, we run a persistently backlogged flow over a 24Mbps link and then
after 3 seconds start a client attempting to stream a 4k video from a server that supports adaptive
bitrate selection. Without \name (a), the video stream experiences highly variable throughput and 
takes 30 seconds to converge to a fair share of the link. In contrast, with \name (b), the video
stream converges to its fair share within 2 seconds and is able to maintain that rate for the
entirety of the stream. This stability provides the best scenario for the ABR algorithm to select
the highest possible bitrate and thus maximize QoE.
}

\paragrapha{Aggregate congestion control is not enough} It is important to note that \name's congestion control by itself (\ie running FIFO scheduling) is not a means of achieving improved performance. 
To see why this is the case, recall that \name does not modify the endhosts: they continue to run the default Cubic congestion controller, which will probe for bandwidth until it observes loss.
Indeed, the packets endhost Cubic sends beyond those that the link can transmit must queue somewhere in the network or get dropped. 
Without \name, they queue at the bottleneck link;
with \name, they instead queue at the \inbox. 
In addition, the delay-based congestion controller at \inbox also maintains a small standing queue at the bottleneck link (which can be seen in Figure~\ref{fig:design:shift-bottleneck}) to avoid under-utilization, which increases the end-to-end-delays slightly. 
Therefore, doing the FIFO scheduling at the \name, as is done by the \baseline, results in slightly worse performance.

% cut, combine with overview-benefits
%\begin{figure}
    \centering
\begin{knitrout}
\definecolor{shadecolor}{rgb}{0.969, 0.969, 0.969}\color{fgcolor}
\includegraphics[width=\maxwidth]{figure/eval:fifo-1} 

\end{knitrout}
    \caption{With FIFO scheduling, the benefits of \name are lost: FCTs are 18\% worse in the median.}
    \label{fig:eval:fifo}
\end{figure}
\newcommand{\overviewBenefitsFifoMedian}{1.96\xspace}
\newcommand{\overviewBenefitsFifoWorse}{18\%\xspace}


\subsection{Impact of Congestion Control} \label{s:robust:cross}\label{s:eval:cc}
How does our choice of congestion control impact our results? 
We first verify that \name indeed runs congestion control algorithms appropriately.
\begin{figure}
    \centering
    \begin{subfigure}[b]{0.5\textwidth}
\begin{knitrout}
\definecolor{shadecolor}{rgb}{0.969, 0.969, 0.969}\color{fgcolor}
\includegraphics[width=\maxwidth]{figure/eval:tput-delay:a-1} 

\end{knitrout}
    \caption{Throughput (in Mbit/s) Comparison.}\label{fig:eval:tputdelay:a}
    \end{subfigure}
    \begin{subfigure}[b]{0.5\textwidth}
\begin{knitrout}
\definecolor{shadecolor}{rgb}{0.969, 0.969, 0.969}\color{fgcolor}
\includegraphics[width=\maxwidth]{figure/eval:tput-delay:b-1} 

\end{knitrout}
    \caption{Delay (in ms) Comparison.}\label{fig:eval:tputdelay:b}
    \end{subfigure}

    \caption{\name can shape underlying Cubic flows so they assume the characteristics of \name's congestion control algorithm.}
    \label{fig:eval:tputdelay}
\end{figure}

Figure~\ref{fig:eval:tputdelay} shows that for two of the three congestion control algorithms we evaluate --- Copa~\cite{copa} and Nimbus~\cite{nimbus} --- the throughput (in Figure~\ref{fig:eval:tputdelay:a}) and queueing delay (in Figure~\ref{fig:eval:tputdelay:b}) distributions over a one-minute experiment are similar to the same algorithm running at the end-host (\ie without \name).

Furthermore, \name is compatible with multiple endhost congestion control algorithms.
When we configure endhosts to use BBR\footnote{We use the BBR implementation provided in Linux $4.13$.} (as opposed to Cubic, as above), \name's benefits remain: \name achieves 58\% lower FCTs in the median.
This is primarily because in the \baseline using BBR causes endhosts to achieve 66\% worse median slowdown ($1.62$ with Cubic to $2.68$ with BBR); \name's slowdown is only 5\% worse when endhosts use BBR ($1.08$ with Cubic to $1.14$ with BBR).

BBR's~\cite{bbr} delay distribution suffers because of its interaction with the end-to-end congestion control. When BBR enters its \texttt{PROBE\_RTT} mode by setting a congestion window of $4$ packets, end-host implementations simply stop transmitting.
However, with \name the end-host implementation is Cubic; it continues probing for bandwidth until the \inbox is forced to drop packets.

Furthermore, correct congestion control behavior is crucial to achieving low FCTs.
In Figure~\ref{fig:eval:cc} we compare three congestion control protocols to the Baseline: BBR~\cite{bbr}, Nimbus~\cite{nimbus}, and Copa~\cite{copa}.
In this scenario, it is important to control delays in the bottleneck queue, since it is FIFO scheduled and therefore queued packets from short requests must wait behind those from longer requests. Nimbus and BBR both maintain slightly higher queueing delays at the bottleneck link, and thus they achieve higher median FCTs \an{numbers}. 
Nimbus, which is slightly more aggressive than Copa, induces a higher queue build up at the bottleneck, and a smaller queue build up at the \name, when compared to Copa. It therefore results in a higher median FCT than Copa, though still providing significant benefits over the baseline. 
BBR, however is even more aggressive and cannot maintain sufficient queuing at the \name to provide enough benefits.

\radhika{Also, no way to further improve BBR?}
\an{try to understand bbr results}

%\radhika{might be interesting to have another line that shows what happens when \name uses Cubic, and how it can break things.}
%\an{I don't think we can run a meaningful cubic experiment since the implementation looks carefully at measurements (loss, out-of-order deliveries) that we don't precisely measure}
%\an{try running a cubic experiment}

\begin{figure}
    \centering
\begin{knitrout}
\definecolor{shadecolor}{rgb}{0.969, 0.969, 0.969}\color{fgcolor}
\includegraphics[width=\maxwidth]{figure/eval_cc-1} 

\end{knitrout}
    \caption{Choosing a congestion control algorithm at \name remains important, just as it is at the end-host. Note the different y-axis scales for each group of request sizes.}
    \label{fig:eval:cc}
\end{figure}
\newcommand{\ccCopaMedian}{1.09\xspace}
\newcommand{\ccNimbusMedian}{1.32\xspace}
\newcommand{\ccBBRMedian}{1.91\xspace}
\newcommand{\ccBaselineMedian}{1.76\xspace}


Meanwhile, the characteristics of other traffic --- not part of the traffic aggregate \name controls --- on the link can force \name's congestion controller to behave more aggressively in order to remain competitive in the bottleneck link. This would reduce the amount of queue build up at the \name, thus reducing its benefits.

\begin{figure}
    \centering
\begin{knitrout}
\definecolor{shadecolor}{rgb}{0.969, 0.969, 0.969}\color{fgcolor}
\includegraphics[width=\maxwidth]{figure/robust_cr-inelastic-1} 

\end{knitrout}
    \caption{Against cross traffic comprising of short lived flows. \name offers 48Mbps of load to the bottleneck queue. The cross traffic's offered load increases along the x-axis, while \name{}'s offered load remains fixed.}
    \label{fig:robust:cr-inelastic}
\end{figure}

\paragrapha{Inelastic Cross Traffic} We first consider the case of \emph{inelastic} cross traffic; that is, traffic that does not respond to queue-size fluctuations.
For example, traffic primarily comprised of short web requests has the inelastic property because regardless of what \name's congestion controller (or any end-to-end congestion controller) does, the component short requests, which remain in TCP slow start for their entirety, will occupy some fraction of the bottleneck link capacity.
In this case, the congestion controller must yield bandwidth to the cross traffic, but can still maintain low delays at the bottleneck link.

We can see this phenomenon in action in Figure~\ref{fig:robust:cr-inelastic}. 
Each of the congestion control algorithms we evaluate sacrifice (to varying degrees) bandwidth in reaction to the cross traffic, which hurts the FCTs of the larger requests.
However, the scheduling policy apportions the remaining bandwidth to the short flows, so there is still an improvement in FCTs at the median.

\begin{figure}
    \centering
\begin{knitrout}
\definecolor{shadecolor}{rgb}{0.969, 0.969, 0.969}\color{fgcolor}
\includegraphics[width=\maxwidth]{figure/robust:cr-elastic-1} 

\end{knitrout}
    \caption{Varying number of competing elastic cross traffic flows. As before, \name's traffic offers 84Mbps of load, with one persistently backlogged connection.}
    \label{fig:robust:cr-elastic}
\end{figure}

\paragrapha{Elastic Cross Traffic} Elastic cross traffic, which fills the bottleneck link's buffer, is the worst-case scenario for \name.
The congestion controller must push packets into the bottleneck queue to compete fairly, and thus it cannot retain packets at the \inbox to schedule.
As a result, we expect performance to be no better than the baseline.
We indeed see this in Figure~\ref{fig:robust:cr-elastic}.

\an{In this experiment, Copa, primarily a delay-based algorithm, cannot adequately detect the presence of competing elastic traffic and ``mode-switch'' to its competitive mode. When we modify Copa to use Nimbus's elasticity detector (Figure~\ref{fig:robust:cr-elastic:b}), its performance matches the baseline.}

\begin{figure}
    \centering
\begin{knitrout}
\definecolor{shadecolor}{rgb}{0.969, 0.969, 0.969}\color{fgcolor}\begin{kframe}


{\ttfamily\noindent\bfseries\color{errorcolor}{\#\# Error in ggproto(NULL, super, call = match.call(), aesthetics = aesthetics, : argument is missing, with no default}}\end{kframe}
\end{knitrout}
    \caption{Competing traffic bundles. Each bundle observes improved median FCT compared to its performance in the baseline scenario.}
    \label{fig:robust:twobundler}
\end{figure}

\paragrapha{Competing Bundles} \name's improvements return once competing traffic starts using \name as well. In Figure~\ref{fig:robust:twobundler}, we show the performance of each of two bundles of traffic competing in the same bottleneck link. 
Despite both bundles containing persistently backlogged flows, just as in Figure~\ref{fig:robust:cr-elastic}, here \name improves the FCTs of both bundles independently.

\subsection{Path Characteristics}\label{s:robust:path}
\an{should these go in ``microbenchmarks'' in \S\ref{s:measurement}?} \radhika{i don't think so}
\begin{outline}
\1 Different RTTs
\1 Different bandwidths
\end{outline}


\cut{
\begin{Appendix}
\section{Varying Offered Load}\label{s:eval:offeredload}
Naturally, if a link is less congested, scheduling the packets that traverse it will have less benefit. Accordingly, as the offered load is reduced, we would expect the gains from scheduling to diminish. 
We now use the web request distribution described in \S\ref{s:eval:setup} to generate a load of 50\% ($48$Mbps), 75\% ($72$Mbps) and 87.5\% ($84$Mbps) of the bottleneck link bandwidth. Our results in Figure~\ref{fig:eval:offeredload} show that as the offered load is decreased, the benefits of \name reduce. This is because if a link is less congested, scheduling the packets that traverse it will have less benefit.
% At 87.5\% load, even without the load offered by a persistently backlogged connection, \name improves FCTs by \an{amount}. 
% As the offered load decreases to 50\%, the benefit provided by \name decreases as well -- to \an{amount} at the 75th percentile.
\end{Appendix}
}

\subsection{Terminating TCP Connections}\label{s:eval:proxy}

Although our \name prototype does not terminate connections (as discussed in \S\ref{s:design:whichcc}), we note that terminating connections does provide one key advantage: the end-to-end congestion controller will observe a smaller RTT, since the proxy can acknowledge its segments much faster than the original receiver. 
This enables rapid window growth at the endhosts.
While there are, of course, operational concerns with managing the resulting queue, it does provide additional scheduling opportunities as well as faster ramp-up for midsized connections.

How much benefit, then, could a proxy-based \name provide?
To evaluate this, we emulate an idealized TCP proxy by modifying the endhosts to maintain a constant congestion window of $450$ packets---slightly larger than the bandwidth-delay product in our setup---and increasing the buffering at the \inbox to hold these packets. 
The other aspects of \name remain unchanged.
The result is in Figure~\ref{fig:eval:proxy}. 

For the short requests which never leave TCP slow start, terminating TCP connections does not yield additional benefits: with or without termination, they finish in a few RTTs.
For medium-to-long requests, terminating TCP connections yields additional benefits since they no longer incur the penalty of window growth.
Therefore, a site may benefit from proxying TCP connections at \name if its traffic pattern contains many medium-sized flows which benefit from fast ramp-up.

\begin{figure}
    \centering
\begin{knitrout}
\definecolor{shadecolor}{rgb}{0.969, 0.969, 0.969}\color{fgcolor}
\includegraphics[width=\maxwidth]{figure/eval:proxy-1} 

\end{knitrout}
    \caption{A proxy-based implementation of \name could yield further benefits to the long flows. Note the different y-axis scales for each group of request sizes.}
    \label{fig:eval:proxy}
\end{figure}


\subsection{Multipath Detection}\label{s:eval:ecmp}

% TODO: after deadline should also run experiment where there are > 1 queues but no imbalance, and then suddenly a large flow comes in and creates imbalance, then leaves, heuristic should go above threshold and come back down correspondingly 
As described in \S\ref{s:queue-ctl:ecmp}, when the ratio of out-of-order to in-order measurements is above a certain threshold, it indicates that \name's component flows are likely traversing multiple imbalanced paths. To evaluate the extent to which this heuristic corresponds with imbalance, we re-run the emulation experiment from Figure~\ref{fig:eval:bigexp} for a variety of network conditions (bottleneck bandwidth ranging from 12 to 96 Mbps, end-to-end RTTs ranging from 10 to 300 ms, and bottleneck load-balancing from 1 to 32 paths) and consider the average value reported by the heuristic over each experiment. The maximum value reported across all experiments with a single path was 0.4\%, while the minimum value reported across all experiments with 2-32 paths was 20\%, two orders of magnitude greater. Thus, this heuristic provides a very clear separation between single and multiple path scenarios and a simple threshold is sufficient. 

% Motivated by these results, we use 0.05 as a threshold for our real-world experiments in the following section. We classified each path as single or multipath by whether or not the component flows experienced different min-RTTs (which we can do in this experiment since we control the flows but \name would not be able to do itself), and found that our chosen threshold correctly distinguished between single and multiple path in all cases. 

% \begin{figure}
%     \centering
% \includegraphics[width=\maxwidth]{figure/eval:ecmp:ratio} 
%     \caption{The queue imbalance heuristic as a function of number of queues. }
%     \label{fig:eval:ecmp:ratio}
% \end{figure}
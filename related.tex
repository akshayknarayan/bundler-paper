\section{Related Work}
\label{s:related}

%With our work sitting at the intersection of highly popular topics of scheduling, congestion control and middleboxes, one can produce a large body of past work related to each topic. However, this unique intersection of topics is also what adds significant novelty to our work. 
Our work is at the intersection of congestion control, scheduling, and middleboxes. 
%We identify the most relevant related work in each of these areas below. 

\Para{Aggregating congestion information} A recent proposal~\cite{fivecomps} observes that the majority of today's traffic is owned by a few entities.
%, and that a single entity might have a large number of flows traversing the same bottleneck link. 
This observation implies that we can expect to have a small number of bundles across the Internet, each with large amounts of traffic, resulting in greater scheduling opportunities within a bundle. Note that the proposal uses its observation to highlight the benefits of sharing information across a given customer's traffic when configuring congestion control algorithms at the endhosts. This is orthogonal to our goal of scheduling such traffic by introducing a middlebox without modifying end hosts.


\Para{Packet scheduling} There have been some recent efforts towards enabling the benefits of packet scheduling. PIFO~\cite{pifo} is a programmable priority queue that can be configured to express different scheduling policies at routers. However, not only does it changing the in-network routers, but it also suffers from the issue of an ISP's limited visibility into the traffic to choose desired policies (and limited incentives to enforce them). UPS~\cite{ups} allows different scheduling policies to be expressed from the edge via header initialization, but also requires changing routers.
%However, since queuing still occurs in the middle of the network, it relies on the customers targeting a common global objective, or on the network operators isolating different customers' traffic, both of which are difficult to realize. 
\name provides a solution that does not require any cooperation from downstream networks in other organizations.

In that spirit, \name is closer to OverQoS~\cite{overqos}. Proposed more than a decade ago, OverQoS aimed to provide QoS benefits in the Internet by aggregating and managing traffic at the nodes of an overlay network~\cite{ron}. Given the aggregation opportunities today, where a few content providers generate majority of the Internet traffic~\cite{fivecomps}, we observe that the time for reaping benefits from such designs may have arrived. Recent trends enable a simpler and more deployable approach that \name adopts, where rather than deploying an overlay network, just deploying a middlebox at the sending and the receiving domains can result in significant amounts of traffic aggregation. In addition, \name provides a novel mechanism to \emph{move} the in-network queues, and actively gain the power to schedule across the bundled traffic. 




%\Para{Overlays} OverQoS~\cite{overqos} proposed an extension to overlay-based network architectures~\cite{ron} to deploy QoS independently of endhosts, a mechanism for packet scheduling.
%OverQoS uses loss-based congestion control and therefore cannot shift queues; our approach is complementary.




\Para{Split TCP Proxy} This middlebox splits the TCP connections traversing it, such that the senders observe a smaller round-trip time and can increase their sending rate faster. Since it terminates and re-establishes TCP connections, it requires implementing a full TCP stack, managing per-connection state and intrusively changing the packet headers. Again, \name has orthogonal (yet, complementary) goals, which allow for a light-weight implementation, without requiring any per-connection state (unless needed for a specific scheduling algorithm) or any modifications to the packets traversing it. We show how combining \name's features with a TCP Proxy could lead to further improvements in performance in \S\ref{s:eval:proxy}. 

%\Para{VPN tunneling} 

% \begin{itemize}
%     \item PIFO and UPS
%     \item Rethinking networking for five computers
%     %\item use of priority queuing in private WANs
%     \item TCP Proxy
%     \begin{outline}
%     \1 An existing approach which could fulfill these design objectives is a TCP Proxy.
%     \2 TCP proxies are popular in cellular networks
%     \2 They are usually used to shorten the observed round-trip time of a connection, so it can quickly ramp up its sending rate.
%     \2 They do this by sending acknowledgements to the sender.  
%         \3 This means they must take responsibility for reliable delivery.
%     \2 Because TCP proxies take responsibility for reliability, they must implement full TCP stacks.
%         \3 As a result, they are difficult to implement in hardware and difficult to scale. \an{maybe mention this later, once it's more clear our design is hardware-compatible?}
%     \2 They furthermore remain limited by one-sided measurements.
%     \2 They do not allow fate-sharing:
%         \3 If a TCP proxy fails, as middleboxes often do~\cite{aplomb}, the underlying connection is broken.
% \1 We describe in \S\ref{s:measurement} how a \name can perform precise congestion control measurements without implementing a TCP proxy
%     \2 and how to perform these measurements in a fault-tolerant way that preserves fate-sharing.
%     \end{outline}
%     \item anything else?
% \end{itemize}
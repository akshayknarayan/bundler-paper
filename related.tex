\section{Related Work}
\label{s:related}

%With our work sitting at the intersection of highly popular topics of scheduling, congestion control and middleboxes, one can produce a large body of past work related to each topic. However, this unique intersection of topics is also what adds significant novelty to our work. 
%Our work is at the intersection of congestion control, scheduling, and middleboxes. 
%We identify the most relevant related work in each of these areas below. 

\Para{Aggregating congestion information} A recent proposal~\cite{fivecomps} observes that the majority of today's traffic is owned by a few entities.
This observation implies that there are large bundles in practice, resulting in greater scheduling opportunities within a bundle. Note that the proposal uses its observation to highlight the benefits of sharing information across a given customer's traffic when configuring congestion control algorithms at the endhosts. This is orthogonal to our goal of scheduling such traffic by introducing a middlebox without modifying end hosts.

\Para{Packet scheduling} There have been some recent efforts towards enabling the benefits of packet scheduling. PIFO~\cite{pifo} is a programmable priority queue that can be configured to express different scheduling policies at routers. However, not only does it require changing the in-network routers, but it also suffers from the issue of an ISP's limited visibility into the traffic to choose desired policies (and limited incentives to enforce them). UPS~\cite{ups} allows different scheduling policies to be expressed from the edge via header initialization, but also requires a change to the routers.
%However, since queuing still occurs in the middle of the network, it relies on the customers targeting a common global objective, or on the network operators isolating different customers' traffic, both of which are difficult to realize. 
\name provides a solution that does not require any cooperation from downstream networks in other organizations.

In that spirit, \name is closer to OverQoS~\cite{overqos}. Proposed more than a decade ago, OverQoS aimed to provide QoS benefits in the Internet by aggregating and managing traffic at the nodes of an already-deployed overlay network~\cite{ron}. 
Given the aggregation opportunities today, where a few content providers generate a majority of the Internet traffic~\cite{fivecomps}, we observe that the time for reaping benefits from such designs may have arrived. 
Recent trends enable a simpler and more deployable approach; instead of deploying an overlay network, \name{} proposes deploying a middlebox at the sending and the receiving domains to aggregate sufficient traffic. 
This simplifies the ``rendezvous'' problem, since \name initializes bundles dynamically (\S\ref{s:design}).
In addition, \name provides a novel mechanism to \emph{move} the in-network queues, and gain the power to schedule the bundled traffic. 

\section{Related Work}
\label{s:related}

\begin{itemize}
    \item PIFO and UPS
    \item Rethinking networking for five computers
    %\item use of priority queuing in private WANs
    \item TCP Proxy
    \begin{outline}
    \1 An existing approach which could fulfill these design objectives is a TCP Proxy.
    \2 TCP proxies are popular in cellular networks
    \2 They are usually used to shorten the observed round-trip time of a connection, so it can quickly ramp up its sending rate.
    \2 They do this by sending acknowledgements to the sender.  
        \3 This means they must take responsibility for reliable delivery.
    \2 Because TCP proxies take responsibility for reliability, they must implement full TCP stacks.
        \3 As a result, they are difficult to implement in hardware and difficult to scale. \an{maybe mention this later, once it's more clear our design is hardware-compatible?}
    \2 They furthermore remain limited by one-sided measurements.
    \2 They do not allow fate-sharing:
        \3 If a TCP proxy fails, as middleboxes often do~\cite{aplomb}, the underlying connection is broken.
\1 We describe in \S\ref{s:measurement} how a \name can perform precise congestion control measurements without implementing a TCP proxy
    \2 and how to perform these measurements in a fault-tolerant way that preserves fate-sharing.
    \end{outline}
    \item anything else?
\end{itemize}
\section{Related Work}
\label{s:related}

%With our work sitting at the intersection of highly popular topics of scheduling, congestion control and middleboxes, one can produce a large body of past work related to each topic. However, this unique intersection of topics is also what adds significant novelty to our work. 
Our work sits at the unique intersection of three popular areas: scheduling, congestion control and middleboxes. We identify the most relevant related works in each of these areas below. 

\Para{Enabling packet scheduling in wide-area networks} PIFO~\cite{pifo} is programmable scheduler that can be configured by the network operators to express different policies. Not only does it require a new router hardware, but it also suffers from the issue of limited visibility into customers' traffic to choose appropriate policies. UPS~\cite{ups} goes a step further by allowing different scheduling policies to be expressed from the edge via header initialization, 
%However, since queuing still occurs in the middle of the network, it relies on the customers targeting a common global objective, or on the network operators isolating different customers' traffic, both of which are difficult to realize. 
but it also requires a new router hardware. \name, with its more tractable goals, provides a solution that is far easier to deploy. 

\Para{Aggregating congestion information} A recent proposal~\cite{fivecomps} observes that majority of today's traffic is owned by a few entities.
%, and that a single entity might have a large number of flows traversing the same bottleneck link. 
This observation ties in well with our model, implying that we can expect to have a small number of bundles across the Internet, each with large amounts of traffic, resulting in greater scheduling opportunities within a bundle. Note that the proposal uses its observation to highlight the benefits sharing information across a single customer's traffic when configuring congestion control algorithms at the endhosts. This is orthogonal to our goal of doing effective scheduling across such traffic by introducing a new middlebox as a choke-point, without touching individual endhosts. 

\Para{TCP Proxy} This is the middlebox that can be considered to be most closely related to \name. It splits the TCP connections traversing it, such that the senders observe a smaller round-trip time and the flows ramp up faster. Since it terminates and re-establishes TCP connections, it requires implementing full TCP stacks, managing per-connection state and intrusively changing the packet headers. Again, \name has orthogonal (yet, complementary) goals, which allow for a light-weight implementation, without requiring any per-connection state (unless needed for a specific scheduling algorithm) or any modifications to the packets traversing it. We show how combining \name's features with a TCP Proxy can lead to further improvements in performance in \S\ref{s:eval:proxy}. 

%\Para{VPN tunneling} 

% \begin{itemize}
%     \item PIFO and UPS
%     \item Rethinking networking for five computers
%     %\item use of priority queuing in private WANs
%     \item TCP Proxy
%     \begin{outline}
%     \1 An existing approach which could fulfill these design objectives is a TCP Proxy.
%     \2 TCP proxies are popular in cellular networks
%     \2 They are usually used to shorten the observed round-trip time of a connection, so it can quickly ramp up its sending rate.
%     \2 They do this by sending acknowledgements to the sender.  
%         \3 This means they must take responsibility for reliable delivery.
%     \2 Because TCP proxies take responsibility for reliability, they must implement full TCP stacks.
%         \3 As a result, they are difficult to implement in hardware and difficult to scale. \an{maybe mention this later, once it's more clear our design is hardware-compatible?}
%     \2 They furthermore remain limited by one-sided measurements.
%     \2 They do not allow fate-sharing:
%         \3 If a TCP proxy fails, as middleboxes often do~\cite{aplomb}, the underlying connection is broken.
% \1 We describe in \S\ref{s:measurement} how a \name can perform precise congestion control measurements without implementing a TCP proxy
%     \2 and how to perform these measurements in a fault-tolerant way that preserves fate-sharing.
%     \end{outline}
%     \item anything else?
% \end{itemize}
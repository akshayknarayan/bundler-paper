\section{Discussion}\label{s:discussion}

\paragrapha{Composability} Bundles are naturally \emph{composable}: a subdomain of domain A can deploy its own \name to take control of its fraction of the in-network queues, with domain A's \name enforcing a scheduling policy across the bundled traffic from each subdomain.  
%\radhika{finally at the right place! but needs more exposition.}
%\an{added some exposition below}
For example, a department within an institute may bundle its traffic to a collaborating department in another institute, with the parent institutes bundling the aggregate traffic across multiple departments.
% \name may be useful at \emph{both} levels of aggregation, because different amounts of traffic experience bottlenecks in different locations.
% An individual user could be bottlenecked on a slow Wi-Fi link. 
% This individual's traffic might mix with an institute's department which, as an aggregate, is bottlenecked on the campus's local network.
% After another layer of aggregation, the institute's traffic could be bottlenecked on an interdomain link.
% At each layer of aggregation, it may be beneficial to control the queues at the respective \name.
% Studying this deployment scenario remains future work.

\paragrapha{Scheduling across different bundles at a \inbox} We evaluate benefits of scheduling \emph{within} a bundle. In practice, a given \inbox will see traffic from multiple bundles. Extending different scheduling policies to multiple such bundles can be done trivially.
%Different bundles may have different rates; recent work~\cite{carousel, eifel} has shown it is possible to implement such multi-rate, multi-scheduler datapaths efficiently.

\paragrapha{Rate allocation across different competing bundles} When multiple bundles (belonging to different sending domains) compete at the same bottleneck, \name's congestion control would ensure a fair rate allocation across each of these bundles, irrespective of the amount of traffic in them. It, therefore, provides fairness on per-sending domain basis, as opposed to a per-flow basis, making it more robust to popular end-host strategies such as opening multiple connections to increase bandwidth share. 
